\documentclass[11 pt, twocolumn]{article}
\usepackage[utf8]{inputenc}
\usepackage[top=2cm, bottom=2cm, right=2cm, left=2cm]{geometry}

%distance between the two columns
\setlength{\columnsep}{30pt}

%switch off page numbering
%\pagenumbering{gobble}

\title{\textbf{COMP 551 PROJECT WRITE-UP
\\~\\{\Large COMP 551, Applied Machine Learning
\\McGill University}}}
\author{Anton Gladyr, Saleh Bakhit, Vasu Khanna }
\date{September 2019}

\begin{document}

\maketitle

\begin{abstract}
Summarize the project task and your most important findings. For example, include sentences like “In this project we investigated the performance of linear classification models on two benchmark
datasets”, “We found that the logistic regression approach was achieved worse/better accuracy than LDA and was
significantly faster/slower to train.” (Summarize the project task, the two datasest, and your most important findings. This
should be similar to the abstract but more detailed. You should include background information and citations to
relevant work (e.g., other papers analyzing these datasets).)
\end{abstract}

\section{Introduction}
In this miniproject you will implement two linear classification techniques—logistic regression and linear discriminant analysis (LDA)—and run these two algorithms on two distinct datasets. These two algorithms are discussed in Lectures 4 and 5, respectively. The goal is to gain experience implementing these algorithms from scratch and to get hands-on experience comparing their performance.

\section{Datasets}
Very briefly describe the and how you processed them. Describe the new features you come up with in detail. Highlight any possible ethical concerns that might arise when working these kinds of datasets. Note: You do not need to explicitly verify that the data satisfies the i.i.d. assumption (or any of the other formal assumptions for linear regression).

\section{Results}
Describe the results of all the experiments mentioned in
Task 3 (at a minimum) as well as any other interesting results you find. At a minimum you must report:
1. A discussion of how the logistic regression performance (e.g., convergence speed) depends on the learning rate.
(Note: a figure would be an ideal way to report these results).
2. A comparison of the runtime and accuracy of LDA and logistic regression on both datasets.
3. Results demonstrating that the feature subset and/or new features you used improved performance.

\section{Discussion and Conclusion}
Summarize the key takeaways from the project and possibly directions for future investigation.

\section{Statement of Contributions}
State the breakdown of the workload across the team members.

\bibliographystyle{plain}
\bibliography{references}

\end{document}
